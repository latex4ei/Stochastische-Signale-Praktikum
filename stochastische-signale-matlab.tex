% % % % % % % % % % % % % % % % % % % % % % % % % % % % % % % % % % % % % % % % 
% Formelsammlung von LaTeX4EI									
%
% @encode: 	UTF-8, tabwidth = 4, newline = LF
% @author:	Lukas Kompatscher
% @date:		
%
% % % % % % % % % % % % % % % % % % % % % % % % % % % % % % % % % % % % % % % % 

%-------------------------------------------%
%  Stochastische Signale MATLAB Praktikum	%
%~~~~~~~~~~~~~~~~~~~~~~~~~~~~~~~~~~~~~~~~~~~%

% Document Class ===============================================================
\documentclass[english]{latex4ei/latex4ei_sheet}

% set document information
\title{Stochastische \\ Signale MATLAB}
\author{LaTeX4EI}							% optional, delete if unchanged
\myemail{lukas.kompatscher@tum.de}			% optional, delete if unchanged


\usepackage{listings}
\usepackage{color} %red, green, blue, yellow, cyan, magenta, black, white
\definecolor{mygreen}{RGB}{28,172,0} % color values Red, Green, Blue
\definecolor{mylilas}{RGB}{170,55,241}

\lstset{language=Matlab,%
	%basicstyle=\color{red},
	breaklines=true,%
	morekeywords={matlab2tikz},
	keywordstyle=\color{blue},%
	morekeywords=[2]{1}, keywordstyle=[2]{\color{black}},
	identifierstyle=\color{black},%
	stringstyle=\color{mylilas},
	commentstyle=\color{mygreen},%
	showstringspaces=false,%without this there will be a symbol in the places where there is a space
%	numbers=left,
	numberstyle={\tiny \color{black}},% size of the numbers
	numbersep=9pt, % this defines how far the numbers are from the text
	emph=[1]{for,end,break},emphstyle=[1]\color{red}, %some words to emphasise
	%emph=[2]{word1,word2}, emphstyle=[2]{style},    
}


% DOCUMENT_BEGIN ===============================================================
\begin{document}

\maketitle

% SECTION ======================================================================
\section{Allgemeine Befehle}
% ==============================================================================

\begin{sectionbox}
	\subsection{Standardbefehle}
	\begin{tablebox}{ll}
		Befehl & Funktion\\ \cmrule
		save(\textit{filename, variable}) & speichert \textit{variable} in matfile\\
		load(\textit{filename}) & lädt Variable aus matfiel \\
		clear \textit{variable} & löscht \textit{variable}\\
		clear all & löscht alle Variablen im Workspace\\
		clc & löscht Inhalt des Kommandofensters\\
		doc \textit{expression} & Hilfedatei zu \textit{expression}\\
		help \textit{expression} & Kurzhilfe zu \textit{expression} \\

	\end{tablebox}
\end{sectionbox}

\begin{sectionbox}
	\subsection{Datentypkonvertierung (Karsten)}
	\begin{tablebox}{ll}
		
		Befehl & Funktion \\\cmrule
		double(\textit{array}) & Umwandlung von \textit{array} in double\\
		
	\end{tablebox}
\end{sectionbox}

\begin{sectionbox}
	\subsection{Allgemeine Rechenoperationen}
	\begin{tablebox}{ll}
		
		Befehl & Funktion \\\cmrule
		mod(x,y) & x modulo y (immer positiv)\\
		rem(x,y) & x modulo y (vorzeichenabhängig)\\
		sqrt(x) & $ \sqrt{x}$\\
		floor(x) & Abrunden auf Integer\\
		ceil(x) & Aufrunden auf Integer\\
		sum(x) & Summe über Werte des Vektors x\\
		prod(x) & Produkt über Werte des Vektors x\\
		min(x) & kleinster Wert des Vektors x\\
		max(x) & größter Wert des Vektors x\\
		all(x) & 1 für keine 0 in Vektor x\\
		any(x) & 1 für eine Nicht-0 in Vektor x\\
		
	\end{tablebox}
\end{sectionbox}

\begin{sectionbox}
	\subsection{Trigonometrische Funktionen}
	\begin{tablebox}{ll}
		
		Befehl & Funktion \\\cmrule
		sin(x) , cos(x), tan(x) & x in Bogenmaß!\\
		sind(x), cosd(x), tand(x) & x in Grad!\\
		asin(x), acos(x), atan(x) & Arcusfunktionen (Rad)\\
		asind(x), acosd(x), antans(x) & Arcusfunktionen (Grad)\\
		
	\end{tablebox}
\end{sectionbox}


\begin{sectionbox}
	\subsection{Komplexe Zahlen}
	\begin{tablebox}{ll}
		
		Befehl & Funktion \\\cmrule
		complex(a,b) & $a+jb$ \\
		real(z) & Realteil von z\\
		imag(z) & Imaginärteil von z\\
		abs(z) & Betrag/Komplexe Amplitude von z\\
		angle(z) & Phase von z\\
		conj(z) & konjugiert komplex von z\\
		
	\end{tablebox}
\end{sectionbox}

\section{Matrizzenrechnung}
\begin{sectionbox}
	\subsection{Rechenoperationen}
	\begin{tablebox}{ll}
		Befehl & Funktion \\\cmrule
		{[}a b c{]} & Zeilenvektor\\
		{[}a; b; c{]} & Spaltenvektor\\
		{[}a b c; d e f; g h i{]} & 3x3-Matrix\\
		inv($\ma A$) & inverse Matrix von $\ma A$\\
		$\ma A$' & $\ma A^\top$\\
		$\ma A\setminus \vec b $ & löst $\ma A \vec x= \vec b$\\
		$\ma A$(m,n) & Element $\ma A_{m,n}$\\
		$\ma A$(m,:) & m. Zeile\\
		$\ma A$(:,n) & n. Spalte\\
		find($\ma A$) & lokalisiert Nicht-Null-Elemente (Indizes)\\
		det($\ma A$) & Determinante von A \\
		a:b:c & Vektor von a bis c mit Schrittweite b\\
		linspace(a,b,n) & n Werte im gleichen Abstand von a bis b\\
		norm(x) & eukl. Norm des Vektors x\\
		$[\ma L \ma R \ma P ] = \text{lu}(\ma A)$ &(LR-) Zerlegung von A in Dreiecksmatrizen \\
		$[\ma Q \ma R] = \text{qr}(\ma A)$& QR-Zerlegung von A\\
	\end{tablebox}
	
	Komponentenweises Rechnen durch einen Punkt vor einem Operator\\
	Bsp: $\ma A$.\^{}2 quadriert jedes Element der Matrix $\ma A$\\
	Inlinefunktion: @(x)(f(x))\\
\end{sectionbox}

\begin{sectionbox}
	\subsection{Spezielle Matrizzen}
	\begin{tablebox}{ll}
		Befehl & Funktion \\\cmrule
		eye(m,n) & mxn Einheitsmatrix\\
		zeros(m,n) & mxn 0-Matrix\\
		ones(m,n) & mxn 1-Matrix\\
		diag([a b]) & Diagonalmatrix mit [a b] auf Diagonale\\
		rand(m,n) & mxn Zufallsmatrix (Werte: 0-1)\\
		randi(imax,m,n) & integer Zufallsmatrix mit max. imax\\
		
	\end{tablebox}
\end{sectionbox}

\section{Schleiflab}
\begin{sectionbox}
	\begin{tablebox}{lr}
		while: & for: \\ \trule
		\emph{while} expression & \emph{for} i=0:1:20\\
		statements & statements \\
		\emph{end} & \emph{end}\\
	\end{tablebox}
	Schleife vorzeitig verlassen mit break
\end{sectionbox}


%=======================================================================
\section{Plot}
%=======================================================================

\subsection{2D Plots}

\begin{sectionbox}
	\begin{lstlisting}
	figure(1);				% new figure
	clf;					% clear old figures
	plot(x, y, 'k');		% plot y(x) in black 'k'
	hold on;				% more plots in same figure
	plot(x, z, 'ro')		% plot z(x) in red circles
	legend('y', 'z')		% names of plots
	hold off;
	\end{lstlisting}
\end{sectionbox}

% SECTION ======================================================================
\section{Woche 1}
% ==============================================================================
\begin{sectionbox}
	\subsection{c Programming Language}

	\lstinputlisting{my_pdf.m}
\end{sectionbox}



% DOCUMENT_END =================================================================
\end{document}
