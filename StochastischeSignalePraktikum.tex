% % % % % % % % % % % % % % % % % % % % % % % % % % % % % % % % % % % % % % % % 
% Formelsammlung von LaTeX4EI									
%
% @encode: 	UTF-8, tabwidth = 4, newline = LF
% @author:	Lukas Kompatscher
% @date:		
%
% % % % % % % % % % % % % % % % % % % % % % % % % % % % % % % % % % % % % % % % 

%-------------------------------------------%
%  Stochastische Signale MATLAB Praktikum	%
%~~~~~~~~~~~~~~~~~~~~~~~~~~~~~~~~~~~~~~~~~~~%

% Document Class ===============================================================
\documentclass[deutsch]{latex4ei/latex4ei_sheet}

% set document information
\title{Stochastische \\ Signale Praktikum}
\author{LaTeX4EI}			
\myemail{lukas.kompatscher@tum.de}		


\usepackage{listings}
\usepackage{color} %red, green, blue, yellow, cyan, magenta, black, white
\definecolor{mygreen}{RGB}{28,172,0} % color values Red, Green, Blue
\definecolor{mylilas}{RGB}{170,55,241}

\lstset{language=Matlab,%
	%basicstyle=\color{red},
	breaklines=true,%
	morekeywords={matlab2tikz},
	keywordstyle=\color{blue},%
	morekeywords=[2]{1}, keywordstyle=[2]{\color{black}},
	identifierstyle=\color{black},%
	stringstyle=\color{mylilas},
	commentstyle=\color{mygreen},%
	showstringspaces=false,%without this there will be a symbol in the places where there is a space
%	numbers=left,
	numberstyle={\tiny \color{black}},% size of the numbers
	numbersep=9pt, % this defines how far the numbers are from the text
	emph=[1]{for,end,break},emphstyle=[1]\color{red}, %some words to emphasise
	%emph=[2]{word1,word2}, emphstyle=[2]{style},    
}


% DOCUMENT_BEGIN ===============================================================
\begin{document}

\maketitle

% SECTION ======================================================================
\section{Allgemeine Befehle}
% ==============================================================================

\begin{sectionbox}
	\subsection{Standardbefehle}
	\begin{tablebox}{ll}
		Befehl & Funktion\\ \cmrule
		save(\textit{filename, variable}) & speichert \textit{variable} in matfile\\
		load(\textit{filename}) & lädt Variable aus matfiel \\
		clear \textit{variable} & löscht \textit{variable}\\
		clear all & löscht alle Variablen im Workspace\\
		clc & löscht Inhalt des Kommandofensters\\
		doc \textit{expression} & Hilfedatei zu \textit{expression}\\
		help \textit{expression} & Kurzhilfe zu \textit{expression} \\

	\end{tablebox}
\end{sectionbox}

\begin{sectionbox}
	\subsection{Datentypkonvertierung (Karsten)}
	\begin{tablebox}{ll}
		
		Befehl & Funktion \\\cmrule
		double(\textit{array}) & Umwandlung von \textit{array} in double\\
		
	\end{tablebox}
\end{sectionbox}

\begin{sectionbox}
	\subsection{Allgemeine Rechenoperationen}
	\begin{tablebox}{ll}
		
		Befehl & Funktion \\\cmrule
		mod(x,y) & x modulo y (immer positiv)\\
		rem(x,y) & x modulo y (vorzeichenabhängig)\\
		sqrt(x) & $ \sqrt{x}$\\
		floor(x) & Abrunden auf Integer\\
		ceil(x) & Aufrunden auf Integer\\
		sum(x) & Summe über Werte des Vektors x\\
		prod(x) & Produkt über Werte des Vektors x\\
		min(x) & kleinster Wert des Vektors x\\
		max(x) & größter Wert des Vektors x\\
		all(x) & 1 für keine 0 in Vektor x\\
		any(x) & 1 für eine Nicht-0 in Vektor x\\
		mean(x) & Mittelwert des Vektors x
		
	\end{tablebox}
\end{sectionbox}

\begin{sectionbox}
	\subsection{Trigonometrische Funktionen}
	\begin{tablebox}{ll}
		
		Befehl & Funktion \\\cmrule
		sin(x) , cos(x), tan(x) & x in Bogenmaß!\\
		sind(x), cosd(x), tand(x) & x in Grad!\\
		asin(x), acos(x), atan(x) & Arcusfunktionen (Rad)\\
		asind(x), acosd(x), antans(x) & Arcusfunktionen (Grad)\\
		
	\end{tablebox}
\end{sectionbox}


\begin{sectionbox}
	\subsection{Komplexe Zahlen}
	\begin{tablebox}{ll}
		
		Befehl & Funktion \\\cmrule
		complex(a,b) & $a+jb$ \\
		real(z) & Realteil von z\\
		imag(z) & Imaginärteil von z\\
		abs(z) & Betrag/Komplexe Amplitude von z\\
		angle(z) & Phase von z\\
		conj(z) & konjugiert komplex von z\\
		
	\end{tablebox}
\end{sectionbox}

\section{Matrizzenrechnung}
\begin{sectionbox}
	\subsection{Rechenoperationen}
	\begin{tablebox}{ll}
		Befehl & Funktion \\\cmrule
		{[}a b c{]} & Zeilenvektor\\
		{[}a; b; c{]} & Spaltenvektor\\
		{[}a b c; d e f; g h i{]} & 3x3-Matrix\\
		{[}a b{]} = size(\ma A) & Dimensionen der Matrix (a Zeilen, b Spalten)\\
		inv($\ma A$) & inverse Matrix von $\ma A$\\
		$\ma A$' & $\ma A^\top$\\
		$\ma A\setminus \vec b $ & löst $\ma A \vec x= \vec b$\\
		$\ma A$(m,n) & Element $\ma A_{m,n}$\\
		$\ma A$(m,:) & m-te Zeile\\
		$\ma A$(:,n) & n-te Spalte\\
		find($\ma A$) & lokalisiert Nicht-Null-Elemente (Indizes)\\
		det($\ma A$) & Determinante von A \\
		a:b:c & Vektor von a bis c mit Schrittweite b\\
		linspace(a,b,n) & n Werte im gleichen Abstand von a bis b\\
		norm(x) & eukl. Norm des Vektors x\\
		$[\ma L \ma R \ma P ] = \text{lu}(\ma A)$ &(LR-) Zerlegung von A in Dreiecksmatrizen \\
		$[\ma Q \ma R] = \text{qr}(\ma A)$& QR-Zerlegung von A\\
	\end{tablebox}
	
	Komponentenweises Rechnen durch einen Punkt vor einem Operator\\
	Bsp: $\ma A$.\^{}2 quadriert jedes Element der Matrix $\ma A$\\
	Inlinefunktion: @(x)(f(x))\\
\end{sectionbox}

\begin{sectionbox}
	\subsection{Spezielle Matrizzen}
	\begin{tablebox}{ll}
		Befehl & Funktion \\\cmrule
		eye(m,n) & $m\times n$ Einheitsmatrix\\
		zeros(m,n) & $m\times n$ 0-Matrix\\
		ones(m,n) & $m\times n$ 1-Matrix\\
		diag([a b]) & Diagonalmatrix mit [a b] auf Diagonale\\
		rand(m,n) & $m\times n$ Zufallsmatrix (Werte: 0-1)\\
		randi(imax,m,n) & integer Zufallsmatrix mit max. imax\\
		
	\end{tablebox}
\end{sectionbox}

\section{Schleiflab}
\begin{sectionbox}
	\begin{tablebox}{lr}
		while: & for: \\ \cmrule
		\emph{while} expression & \emph{for} i=0:1:20\\
		statements & statements \\
		\emph{end} & \emph{end}\\
	\end{tablebox}
	Schleife vorzeitig verlassen mit break
\end{sectionbox}


%=======================================================================
\section{Plot}
%=======================================================================

\subsection{2D Plots}

\begin{sectionbox}
	\begin{lstlisting}
	figure(1);				% new figure
	clf;					% clear old figures
	plot(x, y, 'k');		% plot y(x) in black 'k'
	hold on;				% more plots in same figure
	plot(x, z, 'ro')		% plot z(x) in red circles
	legend('y', 'z')		% names of plots
	hold off;
	\end{lstlisting}
\end{sectionbox}

% SECTION ======================================================================
\section{Stochastische Zufallsvariablen}
% ==============================================================================

\begin{sectionbox}
	\subsection{Realisierung von Standardmodellen}
	\begin{tablebox}{ll}
		Befehl & $m\times n$-Realisierung einer \dots \\\cmrule
		rand(m,n) & gleichverteile ZV (Werte: 0-1)\\
		randn(m,n) & Standardnormalverteilung $\mathcal N(0, 1)$\\
		gamrnd(k,t,m,n) & gammaverteile ZV, shape k, scale t\\
		binornd(n,p,m,n) &  Binomialverteilung mit Parameter n, p\\
		binornd(1,p,m,n) &  Bernoulliverteilung mit Wahrscheinlichkeit p\\
		geornd(p,m,n) & Geometrische Verteilung mit Wahrscheinlichkeit p\\
		exprnd(1/lambda,m,n) & Exponentialverteilung mit Parameter lambda\\
	\end{tablebox}
	\emph{Beispiele:}
	\begin{tablebox}{ll}
		Befehl & Ergebnis \\\cmrule
		2*rand+1 & gleichverteile ZV im Bereich [1,3]\\
		plot(y,unifpdf((y-1)/2)/2) & plottet PDF der oberen ZV\\
		plot(x,unifpdf(2*x)*2) & PDF einer gv. ZV im Bereich [0,0.5]\\
		sigma*randn+mu & Normalverteilung $\mathcal N(\mu, \sigma^2)$
	\end{tablebox}
\end{sectionbox}

\begin{sectionbox}
	\subsection{Wahrscheinlichkeitsdichtefunktion (PDF) $f_{\X}(x)$}
	
	\begin{tablebox}{ll}
		Befehl & PDF einer \dots \\\cmrule
		unifpdf(X,A,B) & Gleichverteilen ZV an X im Intervall [A,B]\\
		normpdf(X,mu,sigma) & Normalverteilung mit Parameter mu und sigma\\
		gampdf(X,A,B) & Gammaverteilung an X mit shape A und scale B\\
		
	\end{tablebox}
	
	\subsection{Kommulative Verteilungsfunktion (CDF) $F_{\X}(x)$}
	\begin{tablebox}{ll}
		Befehl & CDF einer \dots \\\cmrule
		poisscdf(X, lambda) & Poissonverteilung an X mit Parameter lambda\\
		normcdf(X,mu,sigma) & Normalverteilung an X mit Parameter $\mu$ und $\sigma$\\
		cdfplot(X) & Schätzt und plottet CDF von X\\
	\end{tablebox}
\end{sectionbox}


\begin{sectionbox}
	\subsection{Schätzung von Paramtern von Realisierungen einer ZV X}
	x ist ein Vektor von Realisierungen von X
	\begin{tablebox}{ll}
		Befehl & Funktion\\ \cmrule
		mean(x) & Schätzt Erwartungswert $\mu$ der ZV X\\
		var(x) & Schätzt Varianz $\Var(X)$ der ZV X\\
		sqrt(var(x)) & Schätzt $\sqrt{\Var(X)} = \sigma$ der ZV X\\
		lenght(x) & Anzahl der Realisierungen der ZV X\\
		mean(x.\^{}3) & Schätzung des 3. Moments der ZV X
	\end{tablebox}
	
\end{sectionbox}

% SECTION ======================================================================
\section{Kapitel 1}
% ==============================================================================
\begin{sectionbox}
	\subsection{Histogramm einer gleichverteilten ZV plotten}
	\lstinputlisting{matlab/hist_rand.m}
\end{sectionbox}



% SECTION ======================================================================
\section{Kapitel 3: Bedingte Verteilung}
% ==============================================================================
\begin{sectionbox}
	\subsection{Histogramm eines AWGN-Kanals}
	\parbox{3cm}{
	\includegraphics[width = 3cm]{img/awgn-channel.png}
	}
	\parbox{4cm}{
		$\X$ Eingangssignal\\
		$\Z = \mathcal N(0, \sigma^2)$ weißes Rauschen\\
		$\Y$ Ausgangssignal
		}
	\lstinputlisting{matlab/hist_out.m}

	\textbf{Maximum-Likelihood-Detektion}
	\[ \max_{\hat x \in \{0,1\}}f_{\Y|\X}(y|\hat x) \]
	In unserem Fall: $\hat x = \begin{cases}
	0 & y\le \frac{1}{2}\\
	1 & y > \frac{1}{2}
	\end{cases}$
	\begin{lstlisting}[gobble=4]
	function xhat = ml_detector(y)
	xhat=(y>0.5);
	\end{lstlisting}
	\textbf{Maximum-A-Posteriori-Detektion}
	\[\max_{\hat x \in \{0,1\}} p_{\X|\Y}(\hat x|y)\] mit
	\[p_{\X|\Y}(x|y)=\begin{cases}
	 \frac{p f_{\Z}(y-1)}{f_{\Y}(y)} & x=1\\
	\frac{(1-p)f_{\Z}(y)}{f_{\Y}(y)} & x=0\\
	0 & sonst
	\end{cases} 
	\Bigg\} = \frac{f_{\Y|\X}(y|x)p_{X}(x)}{f_{\Y}(y)}\]
	\begin{lstlisting}[gobble=4]
	function xhat = map_detector(y,p,sigma)
	xhat=(p*normpdf(y-1,0,sigma)>(1-p)*normpdf(y,0,sigma));
	\end{lstlisting}
	\vspace{-7pt}
\end{sectionbox}

\section{Kapitel 4: Standardmodelle, Erwartungswert und Varianz}
\begin{sectionbox}
	\subsection{Modellierung für Mobilfunknetz}
	Überlagerung von Nutzern im Hotspot-Bereich (Normalverteilung) $\X_h, \Y_h$ und anderen Nutzern (Gleichverteilung) $\X_h, \Y_h$ mit der Bernoulli-verteilten ZV $B$.
	\[\X_m=\begin{cases}
	\X_h & \text{wenn } B=1\\
	\X_u & \text{wenn } B=0\\
	\end{cases} \qquad 
	\Y_m=\begin{cases}
	\Y_h & \text{wenn } B=1\\
	\Y_u & \text{wenn } B=0\\
	\end{cases}
	\]
	\lstinputlisting{matlab/mixed_positions.m}
\end{sectionbox}

\begin{sectionbox}
	\subsection{Empirische Realisierung von CDFs}	
	\parbox{3.5cm}{
		\includegraphics[width = 3.5cm]{img/cdf-sample.png}
	}
	\parbox{3.5cm}{
		(1)	cdfplot(rand(1000,1)); \\\\
		(2)	cdfplot(sign(rand(1000,1).5)); \\\\
		(3)	cdfplot(randn(1000,1)*3-1.5); \\\\
		(4)	cdfplot(rand(1000,1)*3-1.5);
		}
\end{sectionbox}

\section{Kapitel 5: Zufallsfolgen}
\begin{sectionbox}
	\subsection{Realisierung einer Zufallsfolge}
	$\X_{n+1}=\X_n+V_n \qquad \Y_{n+1}=\Y_n+W_n$ \\
	$V_n$ und $W_n$ sind i.i.d. Gleichverteilung mit Intervall $[-\delta;\delta]$
	\begin{lstlisting}[gobble=4]
	function pos=update_positions(pos,delta)
	% Fuegen Sie hier Ihren Code ein
	pos=pos+2*delta*(rand(size(pos))-0.5);
	\end{lstlisting}
\end{sectionbox}


% DOCUMENT_END =================================================================
\end{document}
